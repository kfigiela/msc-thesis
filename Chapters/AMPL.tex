\chapter{Mathematical programming using AMPL}
\label{chap:ampl} 
\lhead{Chapter \ref{chap:ampl}. \emph{Mathematical programming using AMPL}}

Section \ref{sec:ampl:mathprog} presents overview on mathematical programming and problem classification (Section \ref{sec:ampl:classification}). In section \ref{sec:ampl:ampl} AMPL and other tools for mathematical programming are introduced. Then in Section \ref{sec:ampl:whiskas} we formulate example models: linear \emph{Whiskas Cat Food Problem} and integer \emph{shift work scheduling}.

\section{Mathematical Programming}
\label{sec:ampl:mathprog}

Nowdays, the term programming~\cite{Programming} means writing software, but in 1940s this word was used to describe planning and scheduling activities. It appeared then that restrictions in the planning or scheduling problem may be represented mathematically using equalities and inequalities.  The solution satysfying all these constraints would be considered as acceptable plan or schedule.  Mathematical programming enables us to formally define optimization problem: it's varialbes, objective and constraints.

Defining the problem is not an easy task. If there are too few constraints, the space of possible solutions is too big. On the other hand, too many constraints can rule desirable solutions out. In the worst case there are no solutions at all. The success of programming relies on key insight into the optimized domain and modelling techniques to find a way round the possible difficulty. 

In addition to the constraints, one can define the objective --- function of the variables that makes it possible to compare solutions and select the best one. It doesn't matter how many solutions satisfy the constraints --- we are interested in the one that minimizes or maximizes the objective.

In development of optimization model it is very important to classify the problem, so we can select the most suitable way of solving it. If constraints and objectives are linear combinations of the variables then the model is called \emph{linear program} and the process of modelling and solving is called \emph{linear programming}. This class of optimization problems is particurarly important becasue a lot of real world optimization problems may be represented in such way. Additionally, there exists a lot of theory and algorithms to solve such problems in fast, deterministic way even if they have thousands of variables. The ideas of linear programming are also important for analyzing and solving problems that are non-linear.

All useful methods of mathematical programming involve using computers. The first computional method of solving optimization problems, the simplex algorithm, was introduced at that time and was subject to several improvements over the decades.

\section{Problem classification}
\label{sec:ampl:classification}

In spite of the broad applications of linear programming, the linearity assumption is too unrealistic to be applied to many of real problems. If instead smooth non-linear functions of the variables are used in constraints and objectives we call the program as \emph{non-linear program}. Solving such problems is much harder, but not impossible.

There is also another class of problems called \emph{integer programming} that assumes that variables are integer and in general it is much harder than previous. Fortuanetly, computational power of computers is still increasing and there are efficient algorithms to deal with them.

The optimization problems may be categorized in the following groups:
\begin{description}
  \item[Linear programming (LP)] Objective and constraints in this class are linear functions. Problems in this groups are usually solved by using \emph{simplex}, \emph{interior} or \emph{barrier} method.
  \item[Quadratic programming (QP)] Convex or concave objective and linear constraints. Solved by simplex-type or interior-type method.
  \item[Non-linear programming (NLP)] Continuous, but not all-linear objective and constraints. May be solved by several methods including gradient, quasi-newton, augmented lagrangian and interior-point. Unless special conditions are met, solution found is possibly optimal over only some local neighbourhood. If objective is convex (if minimized) or concave (if maximized) and constraints define a convex region it is guaranteed that optimum found is optimal over the entire feasible region.
  \item[Mixed-integer programming (MIP)] Linear objective and constraints, some or all of variables are integer-valued. Solved by branch-and-bound approach that uses a linear solver to solve subproblems.
  \item[Mixed-integer non-linear programming (MINLP)] Non-linear objective and constraints, some or all of variables are integer-valued. Solved by branch-and-bound approach that uses a non-linear solver to solve subproblems.
  \item[Constraint programming (CP)] In that case, no assumptions can be made on the constraints or the objective. This class is the hardest to solve, as no heuristics can be used. Example problem of this class is \emph{boolean satisfiability problem (SAT)}. Often the problem is simplified to find the feasible solution without any optimization.
\end{description}

\section{AMPL: A Mathematical Programming Language}
\label{sec:ampl:ampl}

To successfully solve optimization problem one needs to do a sequence of multiple tasks as follows:

\begin{enumerate}
  \item Formulate an abstract model: define variables, constraints and objective.
  \item Collect the data for a specific problem instance.
  \item Generate instance-specific variables, constraints and objective.
  \item Solve the problem by running a program called solver that implements algorithm that finds optimal solution.
  \item Analyze the results.
  \item Refine the model and the data as necessary, and repeat.
\end{enumerate}

Unfortuanetly, usually people use diffrent form of representing the data than algorithms do. This makes formulation and generation phases complex as modeller would like to express constraints in human-readable language e.g. mathematical notation, and solvers require to provide multiple matrices as input. We need to transform \emph{modeller's form} to \emph{algorithm's form}. Doing it manually is time consuming and erron-prone task.

To automate this task matrix generators were created for specific models. Altough they successfully automate matrix generation they are hard to code, debug and maintain. Modeller needs to be both mathematician and programmer. The other way to solve that problem is to use mathematical modelling language. Several languages\footnote{i.e. AMPL~\cite{Fourer2002}, Gams~\cite{Gams}, PuLP~\cite{PuLP}, OscaR~\cite{OscaR}} were created over the decades. 

By using modelling language, modeller may express in comfortable way also designed to serve as input for the computer. Then matrix generation may be fully automated without intermediate state of computer programming, thus mathematical programming becomes cheaper and more reliable. Benefits of formulating in modelling languages become particuraly advantageous for models being developed and subject to change.

AMPL is an algebraic mathematical modelling lanuage that resembles traditional mathematical notation to describe variables, objectives and constraints. Code in AMPL will be familiar for anybody that studied basic algebra or calculus, so that he or she doesn't need to be programmer (in present meaning). Algebraic modelling languages allow to express a wide range of optimization problems: linear, nonlinear and integer.

\subsection{Available solvers}

As soon as model is formulated and matrices generated, we may proceed with solving the specific instance of our problem. To do that we will need solver -- a program that implements one of solving algorithms. There is wide range of existing solvers available, both open-source (i.e. Cbc~\cite{cbc-solver}) and commercial (i.e. CPLEX~\cite{cplex}) ones that differ with the problem classes they target. Full list of available solvers is published at AMPL website~\cite{AMPLSolvers}.

Usually solvers provide multiple options that let us tune them for the specific application. One may enable or disable certain features of the solver, i.e. for \emph{Bonmin}~\cite{Bonmin} solver we may choose branching algorithm or configure it to use heuristics.

\section{Example -- Whiskas Cat Food Problem}
\label{sec:ampl:whiskas}
% zaczerpniete mocno z http://twiki.esc.auckland.ac.nz/twiki/bin/view/OpsRes/WhiskasCatFoodProblem

To get some practise with modelling, we will describe a typical linear programming problem on the example of Whiskas Cat Food problem. This is typical planning problem that may be found in many textbooks~\cite{dantzig}. The company producing the food wants to produce it as cheaply as possible while ensuring they meet the stated nutritional analysis requirements stated on the cans. 

Main ingredients of the cat food used are chicken, beef, mutton, rice wheat and gel. The prices for the ingredients are presented in Table \ref{ampl:whiskas:prices}, while ingredient contribution to the total weight of protein, fat, fibre and salt in the final product are give in Table \ref{ampl:whiskas:contribution} and nutritional requirements are presented in Table \ref{ampl:whiskas:analysis}. Given that data we may proceed with model formulation.

\begin{table}
  \centering
  \begin{tabular}{| l | r |}
    \hline
    \textbf{Ingredient} & \textbf{Price per gram} \\ \hline
    Chicken & \$ 0.013 \\ \hline
    Beef & \$ 0.008 \\ \hline
    Mutton & \$ 0.010 \\ \hline
    Rice & \$ 0.002 \\ \hline
    Wheat & \$ 0.005 \\ \hline
    Gel & \$ 0.001 \\ \hline
  \end{tabular}
  \caption{Cat food ingredient's pricing.}
  \label{ampl:whiskas:prices}  
\end{table}

\begin{table}
  \centering
  \begin{tabular}{| l | r | r | r | r |}
    \hline
    & \textbf{Protein} & \textbf{Fat} & \textbf{Fibre} & \textbf{Salt} \\ \hline
    \textbf{Chicken} & 0.100 & 0.080 & 0.001 & 0.002 \\ \hline
    \textbf{Beef} & 0.200 & 0.100 & 0.005 & 0.005 \\ \hline
    \textbf{Mutton} & 0.150 & 0.110 & 0.003 & 0.007 \\ \hline
    \textbf{Rice} & 0.000 & 0.010 & 0.100 & 0.002 \\ \hline
    \textbf{Wheat bran} & 0.040 & 0.010 & 0.150 & 0.008 \\ \hline
    \textbf{Gel} & -- & -- & -- & -- \\ \hline
  \end{tabular}
  \caption{Ingredient contribution to the final product in grams per gram of ingredient.}
  \label{ampl:whiskas:contribution}  
\end{table}

\begin{table}
  \centering
  \begin{tabular}{| l | r |}
    \hline
    Minimum \% Crude Protein & 8.0 \\ \hline
    Minimum \% Crude Fat & 6.0 \\ \hline
    Maximum \% Crude Fibre & 2.0 \\ \hline
    Maximum \% Salt & 0.4 \\ \hline
  \end{tabular}
  \caption{Cat food nutritional analysis.}
  \label{ampl:whiskas:analysis}  
\end{table}


  
\subsection{Problem formulation}

In this particular problem data defines the following data sets:

\begin{itemize}
  \item $I = \left\{\text{chicken}, \text{beef}, \text{mutton}, \text{rice}, \text{wheat}, \text{gel}\right\}$ -- defines possible ingredients,
  \item $C = \left\{\text{protein}, \text{fat}, \text{fibre}, \text{salt}\right\}$ -- defines components of nutrition.
\end{itemize}

We have also some numbers that describe members of sets;
\begin{itemize}
  \item $p_i$ -- price of given ingredient $i$ in \$ per gram
  \item $c_i,c$ -- contribution of ingredient $i$ to component of nutrition $c$ in grams per gram of ingredient.
\end{itemize}

\paragraph{Identify the decision variables}

First of all we need to identify decision variables. For the Whishas Cat Food Problem the descisions are the amounts of each ingredient we put in the can. Formally we could write this as:
\begin{align} 
  x_i &= \text{ amount (g) of ingredient $i$  in a can of cat food}
\end{align} 

\paragraph{Formulate the Objective Function}

The objective for this problem is to minimize the total cost of ingredients per fan of cat food. We know the cost per gram of each ingredient and the amount is to be found.

\begin{align}
   \min \mathop\sum\limits_{i \in I} p_i x_i
\end{align}

\paragraph{Formulate the constraints}

The constraints for the Whiskas Cat Food are:

\begin{enumerate}
  \item The sum of the amounts must make up the whole can (i.e. 100 g).
  \item The stated nutritional analysis requirements are met.
\end{enumerate}

First of the constraints can is: 

\begin{align}
   \mathop\sum\limits_{i \in I} x_i = 100
\end{align}

The latter can be written as follows

\begin{align}
   \mathop\sum\limits_{i \in I} c_{i,protein} x_i \geq 8.0 \\
   \mathop\sum\limits_{i \in I} c_{i,fat} x_i \geq 6.0 \\
   \mathop\sum\limits_{i \in I} c_{i,fibre} x_i \geq 2.0 \\ 
   \mathop\sum\limits_{i \in I} c_{i,salt} x_i \leq 0.4 \\
\end{align}

or in more general way, we may define lower and upper bounds for each component of nutrition as $L_c$ and $U_c$, the values are presented in Table \ref{ampl:whiskas:bounds}. Constraint will be written as

\begin{align}
   \mathop\forall\limits_{c \in C} L_c \leq \mathop\sum\limits_{i \in I} c_{i,c} x_i \leq U_c \\
\end{align}

\begin{table}
  \centering
  \begin{tabular}{| l | r | r |}
    \hline
    \textbf{Component of nutrition} & \textbf{Lower bound} & \textbf{Upper bound} \\ \hline
    Protein & 8.0 & -- \\ \hline
    Fat & 6.0 & -- \\ \hline
    Fibre & 2.0 & -- \\ \hline
    Salt & -- & 0.4 \\ \hline
  \end{tabular}
  \caption{Bounds for contribution of component of nutrition in percent.}
  \label{ampl:whiskas:bounds}  
\end{table}



We have formulated general problem using mathematical notation. Now we will proceed with model problem formulation using AMPL.

\subsection{Problem formulation using AMPL}

AMPL enables us to separate model definition and instance specific data. Usually we create three files: model, data and calling script. In the model file we define the data we need to have: sets and parameters, objective and constraints. Then in data file we populate the sets and parameters with the numbers for the particular instance of the problem. Both model and data files are loaded from calling script that may do some pre or post processing.

\paragraph{Model formulation}

First of all, we should define the sets for the ingredients and components of nutrition. We create file called \texttt{whiskas.mod} that will contain abstract model of optimization: sets, parameters, constraints and objective.

\begin{lstlisting}
set INGREDIENTS;
set COMPONENTS;
\end{lstlisting}

Using sets we can define the decision variables
\begin{lstlisting}
var Amount {INGREDIENTS} >= 0;
\end{lstlisting}

and parameters for the costs, components of nutrition contribution, restrictions and can size.

\begin{lstlisting}
param Cost {INGREDIENTS} >= 0;
param Contribution {INGREDIENTS, COMPONENTS} >= 0;

param Lower {COMPONENTS} default -Infinity;
param Upper {c in COMPONENTS} >= Lower[c], default Infinity;

param CanSize >= 0;

\end{lstlisting}

and the objective function

\begin{lstlisting}
minimize TotalCost : sum {i in INGREDIENTS} Cost[i] * Amount[i];
\end{lstlisting}

Note that, for the bounds that are not set, we assume \emph{±Infinity}. Additionally, we define that \emph{Cost} and \emph{Contribution} parameters should be non negative -- AMPL provides also verification of parameter data.


Finally, we may define constraints

\begin{lstlisting}
subject to MeetRequirements {c in COMPONENTS}:
  Lower[c] <= sum {i in INGREDIENTS} Contribution[i, c] * Amount[i] <= Upper[c];
  
subject to FullCan: 
  sum {i in INGREDIENTS} Amount[i] = CanSize;
\end{lstlisting}

\paragraph{Providing data}

Now we can provide the model with the data. To do this we create the file \texttt{whiskas.dat}.

First of all we need to provide sets we defined in previous paragraph.
\begin{lstlisting}
set INGREDIENTS := CHICKEN BEEF MUTTON RICE WHEAT GEL;
set COMPONENTS  := PROTEIN FAT FIBRE SALT;
\end{lstlisting}

Then we populate parameters with the data.
\begin{lstlisting}
param     Cost :=
  CHICKEN 0.013
  BEEF    0.008
  MUTTON  0.010
  RICE    0.002
  WHEAT   0.005
  GEL     0.001
;

param     Lower :=
  PROTEIN 8.0
  FAT     6.0
;

param     Upper :=
  FIBRE   2.0
  SALT    0.4
;

param CanSize := 100;

param Contribution :
          PROTEIN   FAT FIBRE  SALT :=
  CHICKEN   0.100 0.080 0.001 0.002 
  BEEF      0.200 0.100 0.005 0.005 
  MUTTON    0.150 0.110 0.003 0.007 
  RICE      0.000 0.010 0.100 0.002 
  WHEAT     0.040 0.010 0.150 0.008 
  GEL       0.0   0.0   0.0   0.0   
;
\end{lstlisting} 

\paragraph{Running AMPL}

Now we are ready to solve the model we formulated with AMPL. For that, let's create file called \texttt{whiskas.run}.

First of all we should reset AMPL environment in case that specific AMPL instance was solving other model before. We can do it with command \texttt{reset}. Then we load model using command \texttt{model} that takes model file name as an argument. Next, we load data file using command \texttt{data}. We also need to tell AMPL which solver we would like to be used. In our case we will use CPLEX (\texttt{option solver cplex}). Finally, we call solver and present results. The full file will look like as follows.

\begin{lstlisting}
reset;

model whiskas.mod;

data whiskas.dat;

option solver cplex;
solve;

display Amount;
\end{lstlisting}

Now we may run run AMPL and see if the model is working and if so, what is the optimal solution.

\begin{lstlisting}
% ampl whiskas.run
CPLEX 12.4.0.1: optimal solution; objective 0.52
2 dual simplex iterations (0 in phase I)
Amount [*] :=
   BEEF  60
CHICKEN   0
    GEL  40
 MUTTON   0
   RICE   0
  WHEAT   0
;
\end{lstlisting}

In that case, it appears that it is cheapest to use beef and fill the rest of the can with gel.

\section{Example -- Shift Scheduling}
\label{sec:ampl:sched}

In this section, we will formulate AMPL model for shift scheduling problem~\cite{Fourer2002}. The factory is working in three shift work schedule, operating six days a week from Monday to Saturday. On Saturday there's no third shift as it would overlap with Sunday. There's a certain number of workers required for each shift: more workers are required during first shift and less during the others. Due to work time regulations not all shift combinations of 5 day working week are possible -- in fact that there are only 126 shift schedules possible. The crew wages depend on the choosen schedule. We also want to limit number of different schedules used, so workers may work in teams during planned week. The problem is to minimize the cost of running factory while meeting all regulatory constraints. This is the example of \emph{integer linear programming (ILP)} problem, as only the integer solutions are valid. 
 
\subsection{Problem formulation}

This particular problem defines the following sets: 

\begin{itemize}
  \item $S$ -- set of shifts,  
  \item $W$ -- set of acceptable schedules.
\end{itemize}

We have also some numbers that describe work planning;
\begin{itemize}
  \item $p_w$ -- pay rate at schedule $w$,
  \item $r_s$ -- number of staff required at shift $s$,
  \item $n_{min}$ -- minimal number of workers on any schedule.
\end{itemize}

\paragraph{Identify the decision variables}

We need to decide if given schedule will be used and if so, how many workers should follow it. Formally we could write this as:
\begin{align} 
  u_w &= \text{binary; 1 iif schedule $w$ should be used, otherwise 0}; \\
  n_w &= \text{integer; how many workers should use schedule $w$}.
\end{align} 

\paragraph{Formulate the Objective Function}

The objective for this problem is to minimize the total cost of running factory.

\begin{align}
   \min \mathop\sum\limits_{w \in W} p_w n_w
\end{align}

\paragraph{Formulate the constraints}

The constraints are:

\begin{enumerate}
  \item The stated minimal number of workers on each shift is met.
  \item Each schedule, if used at all, is used by minimal number of workers.
\end{enumerate}

First of the constraints is: 

\begin{align}
   \mathop\forall\limits_{s \in S}
     \mathop\sum\limits_{w \in W: s \in W_w} n_w \geq r_s
\end{align}

The second can be written as follows

\begin{align}
   \mathop\forall\limits_{w \in W}
     n_{min} u_w \leq n_w \\
   \mathop\forall\limits_{w \in W}
     n_w \leq \left(\max\limits_{s \in W_w} r_s\right) u_w 
\end{align}

We have formulated general problem using mathematical notation. Now we will proceed with model problem formulation using AMPL.

\subsection{Problem formulation using AMPL}

After we formulated the abstract model we implement it in AMPL by following the same procedure as in previous section. 

\paragraph{Model formulation}

First of all, we should define the sets and parameters in model file.

\begin{lstlisting}
set SHIFTS;
param Nsched;
set SCHEDS = 1..Nsched;
set SHIFT_LIST {SCHEDS} within SHIFTS;

param rate {SCHEDS} >= 0;
param required {SHIFTS} >= 0;
param least_assign >= 0;
\end{lstlisting}

Using sets we can define the decision variables
\begin{lstlisting}
var Work {SCHEDS} >= 0 integer; 
var Use {SCHEDS} >= 0 binary;
\end{lstlisting}

and the objective function

\begin{lstlisting}
minimize Total_Cost: sum {j in SCHEDS} rate[j] * Work[j];
\end{lstlisting}

Finally, we may define constraints

\begin{lstlisting}
subject to Shift_Needs {i in SHIFTS}:
  sum {j in SCHEDS: i in SHIFT_LIST[j]} Work[j] >= required[i];
subject to Least_Use1 {j in SCHEDS}: 
  least_assign * Use[j] <= Work[j];
subject to Least_Use2 {j in SCHEDS}:
  Work[j] <= (max {i in SHIFT_LIST[j]} required[i]) * Use[j];
\end{lstlisting}

\paragraph{Providing data}

Now we can provide the model with the data. We provide the data we defined in previous paragraph.

\begin{lstlisting}
set SHIFTS := Mon1 Tue1 Wed1 Thu1 Fri1 Sat1
              Mon2 Tue2 Wed2 Thu2 Fri2 Sat2
              Mon3 Tue3 Wed3 Thu3 Fri3 ;

param rate  default 1 ;

param required :=  Mon1 100  Mon2 78  Mon3 52 
                   Tue1 100  Tue2 78  Tue3 52
                   Wed1 100  Wed2 78  Wed3 52
                   Thu1 100  Thu2 78  Thu3 52
                   Fri1 100  Fri2 78  Fri3 52
                   Sat1 100  Sat2 78 ;
                   
param least_assign  := 5;
param Nsched        := 126 ;

set SHIFT_LIST[  1] := Mon1 Tue1 Wed1 Thu1 Fri1 ;
set SHIFT_LIST[  2] := Mon1 Tue1 Wed1 Thu1 Fri2 ;
........
set SHIFT_LIST[126] := Tue2 Wed2 Thu2 Fri2 Sat2 ;
\end{lstlisting} 

\paragraph{Running AMPL}

\begin{lstlisting}
reset;
option omit_zero_rows 1;
model whiskas.mod;

data whiskas.dat;

option solver cplex;
solve;

display Work;
display Total_Cost;
\end{lstlisting}

Now we are ready to run the model and get the results.

\begin{lstlisting}
% ampl sched.run
CPLEX 12.4.0.1: optimal integer solution; objective 266
136 MIP simplex iterations
23 branch-and-bound nodes
Work [*] :=
  3   7
  6  28
 16   8
 18   8
 20  12
 29   7
 37  21
 61   5
 66   5
 78  24
 82  28
 91  12
100   8
102   6
112  28
118  24
122  30
126   5
;

Total_Cost = 266
\end{lstlisting}

Given that data we may assign schedules to the specific workers.

\section{Summary}

This chapter presented basics of mathematical optimization. We discussed briefly history of optimiation, classification of optimization problems and introduced tools for mathematical optimization using computers. We also shown the example problems and corresponding models that may be solved using AMPL.
