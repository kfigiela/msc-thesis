\chapter{Mathematical programming using AMPL}
\label{chap:ampl} 
\lhead{Chapter \ref{chap:ampl}. \emph{Mathematical programming using AMPL}}

Section \ref{sec:ampl:mathprog} presents overview on mathematical programming and problem classification (Section \ref{sec:ampl:classification}). In section \ref{sec:ampl:ampl} AMPL and other tools for mathematical programming are introduced. Then in Section \ref{sec:ampl:whiskas} we formulate simple model using AMPL.

\section{Mathematical Programming}
\label{sec:ampl:mathprog}

\todo{te sekcje przepisac wlasnymi slowami, prostszym jezykiem, rezygnujac z literackich ozdobnikow AMPL-booka, bo tu wygladaja troche nienaturalnie. Oczywiscie dodac cytowania.}

\todo{dodac zdanie, ze programming oznacza tu cos innego niz tworzenie kodu.}
The term \emph{programming} was in use by 1940 to describe planning and scheduling of activities. \emph{Programmers} found that they could represent the amount of each activity using variable that value was to be determined. Then by using equalities and inequalities they were able to mathematically describe restrictions inherent in the planning or scheduling problem. The solution satysfying all these constraints would be considered as acceptable plan or schedule. Mathematical programming enables us to formally define optimization problem: it's varialbes, objective and constraints.

It showed up soon that it was more complex that it initially appeared. It was more than merely specyfying constraints. If there were too few constraints, the space of possible solutions was too big. On the other hand, too many constraints could rule desirable solutions out. In the worst case there were no solutions at all. The success of programming relied on key insight into the optimized domain and modelling techniques to find a way round the possible difficulty. 

In addition to the constraints, one could define the objective --- function of the variables that made it possible to compare solutions and select the best one. Then it didn't matter how many solutions satisfied the constraints --- we were interested in the one that minimized or maximized the objective.

In development of mathematical programming one kind of problem stands out: that in which all the costs, requirements and quantities are proportional to the variables. In other words, constraints and objectives are linear combinations of the variables. Such a program is called \emph{linear program} and the process of modelling and solving is called \emph{linear programming}. This class of optimization problems is particurarly important becasue a lot of real world optimization problems may be represented in such way. Additionally, there exists a lot of theory and algorithms to solve such problems in fast, deterministic way even if they have thousands of variables. The ideas of linear programming are also important for analyzing and solving problems that are non-linear.

All useful methods of mathematical programming involve using computers. It appeared very quickly in 1940s that computers may be used for scientific computing. The first computional method of solving optimization problems, the simplex algorithm, was introduced at that time and was subject to several improvements over the decades.

\section{Problem classification}
\label{sec:ampl:classification}

In spite of the broad applications of linear programming, the linearity assumption is too unrealistic to be applied to many of real problems. If instead smooth non-linear functions of the variables are used in constraints and objectives we call the program as \emph{non-linear program}. Solving such problems is much harder, but not impossible.

There is also another class of problems called \emph{integer programming} that assumes that variables are integer and in general it is much harder than previous. Fortuanetly, computational power of computers is still increasing and there are efficient algorithms to deal with them.

The optimization problems may be categorized in the following groups:
\begin{description}
  \item[Linear programming (LP)] Objective and constraints in this class are linear functions. Problems in this groups are usually solved by using \emph{simplex}, \emph{interior} or \emph{barrier} method.
  \item[Quadratic programming (QP)] Convex or concave objective and linear constraints. Solved by simplex-type or interior-type method.
  \item[Non-linear programming (NLP)] Continuous, but not all-linear objective and constraints. May be solved by several methods including gradient, quasi-newton, augmented lagrangian and interior-point. Unless special conditions are met, solution found is possibly optimal over only some local neighbourhood. If objective is convex (if minimized) or concave (if maximized) and constraints define a convex region it is guaranteed that optimum found is optimal over the entire feasible region.
  \item[Mixed-integer programming (MIP)] Linear objective and constraints, some or all of variables are integer-valued. Solved by branch-and-bound approach that uses a linear solver to solve subproblems.
  \item[Mixed-integer non-linear programming (MINLP)] Non-linear objective and constraints, some or all of variables are integer-valued. Solved by branch-and-bound approach that uses a non-linear solver to solve subproblems.
  \item[Constraint programming (CP)] \emph{TODO: pisac o tym w ogole?}
\end{description}

\section{AMPL: A Mathematical Programming Language}
\label{sec:ampl:ampl}

To successfully solve optimization problem one needs to do a sequence of multiple tasks as follows:

\begin{enumerate}
  \item Formulate an abstract model: define variables, constraints and objective.
  \item Collect the data for a specific problem instance.
  \item Generate instance-specific variables, constraints and objective.
  \item Solve the problem by running a program called solver that implements algorithm that finds optimal solution.
  \item Analyze the results.
  \item Refine the model and the data as necessary, and repeat.
\end{enumerate}

Unfortuanetly, usually people use diffrent form of representing the data than algorithms do. This makes formulation and generation phases complex as modeller would like to express constraints in human-readable language e.g. mathematical notation, and solvers require to provide multiple matrices as input. We need to transform \emph{modeller's form} to \emph{algorithm's form}. Doing it manually is time consuming and erron-prone task.

To automate this task matrix generators were created for specific models. Altough they successfully automate matrix generation they are hard to code, debug and maintain. Modeller needs to be both mathematician and programmer. The other way to solve that problem is to use mathematical modelling language. Several languages\footnote{i.e. AMPL, Gams, PuLP, OscaR}  were created over the decades. 

By using modelling language, modeller may express in comfortable way also designed to serve as input for the computer. Then matrix generation may be fully automated without intermediate state of computer programming, thus mathematical programming becomes cheaper and more reliable. Benefits of formulating in modelling languages become particuraly advantageous for models being developed and subject to change.

AMPL is an algebraic mathematical modelling lanuage that resembles traditional mathematical notation to describe variables, objectives and constraints. Code in AMPL will be familiar for anybody that studied basic algebra or calculus, so that he or she doesn't need to be programmer (in present meaning). Algebraic modelling languages allow to express a wide range of optimization problems: linear, nonlinear and integer.

\subsection{Available solvers}

As soon as model is formulated and matrices generated, we may proceed with solving the specific instance of our problem. To do that we will need solver -- a program that implements one of solving algorithms. There is wide range of existing solvers available, both open-source (i.e. Cbc\footnote{https://projects.coin-or.org/Cbc}) and commercial (i.e. CPLEX\footnote{http://www-01.ibm.com/software/commerce/optimization/cplex-optimizer/}) ones that differ with the problem classes they target. Full list of available solvers is published at AMPL website\footnote{http://www.ampl.com/solvers.html}.

Usually solvers provide multiple options that let us tune them for the specific application. One may enable or disable certain features of the solver, i.e. for \emph{Bonim} solver we may choose branching algorithm or configure it to use heuristics.

\section{Example -- Whiskas Cat Food Problem}
\label{sec:ampl:whiskas}
% zaczerpniete mocno z http://twiki.esc.auckland.ac.nz/twiki/bin/view/OpsRes/WhiskasCatFoodProblem

To get some practise with modelling, we will describe a typical linear programming problem on the example of Whiskas Cat Food problem. The company producing it wants to produce their cat food products as cheaply as possible while ensuring they meet the stated nutritional analysis requirements stated on the cans. 

Main ingredients of the cat food used are chicken, beef, mutton, rice wheat and gel. The prices for the ingredients are presented in Table \ref{ampl:whiskas:prices}, while ingredient contribution to the total weight of protein, fat, fibre and salt in the final product are give in Table \ref{ampl:whiskas:contribution} and nutritional requirements are presented in Table \ref{ampl:whiskas:analysis}. Given that data we may proceed with model formulation.

\begin{table}
  \centering
  \begin{tabular}{| l | r |}
    \hline
    \textbf{Ingredient} & \textbf{Price per gram} \\ \hline
    Chicken & \$ 0.013 \\ \hline
    Beef & \$ 0.008 \\ \hline
    Mutton & \$ 0.010 \\ \hline
    Rice & \$ 0.002 \\ \hline
    Wheat & \$ 0.005 \\ \hline
    Gel & \$ 0.001 \\ \hline
  \end{tabular}
  \caption{Cat food ingredient's pricing.}
  \label{ampl:whiskas:prices}  
\end{table}

\begin{table}
  \centering
  \begin{tabular}{| l | r | r | r | r |}
    \hline
    & \textbf{Protein} & \textbf{Fat} & \textbf{Fibre} & \textbf{Salt} \\ \hline
    \textbf{Chicken} & 0.100 & 0.080 & 0.001 & 0.002 \\ \hline
    \textbf{Beef} & 0.200 & 0.100 & 0.005 & 0.005 \\ \hline
    \textbf{Mutton} & 0.150 & 0.110 & 0.003 & 0.007 \\ \hline
    \textbf{Rice} & 0.000 & 0.010 & 0.100 & 0.002 \\ \hline
    \textbf{Wheat bran} & 0.040 & 0.010 & 0.150 & 0.008 \\ \hline
    \textbf{Gel} & -- & -- & -- & -- \\ \hline
  \end{tabular}
  \caption{Ingredient contribution to the final product in grams per gram of ingredient.}
  \label{ampl:whiskas:contribution}  
\end{table}

\begin{table}
  \centering
  \begin{tabular}{| l | r |}
    \hline
    Minimum \% Crude Protein & 8.0 \\ \hline
    Minimum \% Crude Fat & 6.0 \\ \hline
    Maximum \% Crude Fibre & 2.0 \\ \hline
    Maximum \% Salt & 0.4 \\ \hline
  \end{tabular}
  \caption{Cat food nutritional analysis.}
  \label{ampl:whiskas:analysis}  
\end{table}


  
\subsubsection{Problem formulation}

In this particular problem data defines the following data sets:

\begin{itemize}
  \item $I = \left\{\text{chicken}, \text{beef}, \text{mutton}, \text{rice}, \text{wheat}, \text{gel}\right\}$ -- defines possible ingredients,
  \item $C = \left\{\text{protein}, \text{fat}, \text{fibre}, \text{salt}\right\}$ -- defines components of nutrition.
\end{itemize}

We have also some numbers that describe members of sets;
\begin{itemize}
  \item $p_i$ -- price of given ingredient $i$ in \$ per gram
  \item $c_i,c$ -- contribution of ingredient $i$ to component of nutrition $c$ in grams per gram of ingredient.
\end{itemize}

\paragraph{Identify the decision variables}

First of all we need to identify decision variables. For the Whishas Cat Food Problem the descisions are the amounts of each ingredient we put in the can. Formally we could write this as:
\begin{align} 
  x_i &= \text{ amount (g) of ingredient $i$  in a can of cat food}
\end{align} 

\paragraph{Formulate the Objective Function}

The objective for this problem is to minimize the total cost of ingredients per fan of cat food. We know the cost per gram of each ingredient and the amount is to be found.

\begin{align}
   \min \mathop\sum\limits_{i \in I} p_i x_i
\end{align}

\paragraph{Formulate the constraints}

The constraints for the Whiskas Cat Food are:

\begin{enumerate}
  \item The sum of the amounts must make up the whole can (i.e. 100 g).
  \item The stated nutritional analysis requirements are met.
\end{enumerate}

First of the constraints can is: 

\begin{align}
   \mathop\sum\limits_{i \in I} x_i = 100
\end{align}

The latter can be written as follows

\todo{3.6 powinno byc <, czy raczej w tabelce 3.4 na odwrot?}
\begin{align}
   \mathop\sum\limits_{i \in I} c_{i,protein} x_i \geq 8.0 \\
   \mathop\sum\limits_{i \in I} c_{i,fat} x_i \geq 6.0 \\
   \mathop\sum\limits_{i \in I} c_{i,fibre} x_i \leq 2.0 \\ 
   \mathop\sum\limits_{i \in I} c_{i,salt} x_i \leq 0.4 \\
\end{align}

or in more general way, we may define lower and upper bounds for each component of nutrition as $L_c$ and $U_c$, the values are presented in Table \ref{ampl:whiskas:bounds}. Constraint will be written as

\begin{align}
   \mathop\forall\limits_{c \in C} L_c \leq \mathop\sum\limits_{i \in I} c_{i,c} x_i \leq U_c \\
\end{align}

\begin{table}
  \centering
  \begin{tabular}{| l | r | r |}
    \hline
    \textbf{Component of nutrition} & \textbf{Lower bound} & \textbf{Upper bound} \\ \hline
    Protein & 8.0 & -- \\ \hline
    Fat & 6.0 & -- \\ \hline
    Fibre & 2.0 & -- \\ \hline
    Salt & -- & 0.4 \\ \hline
  \end{tabular}
  \caption{Bounds for contribution of component of nutrition in percent.}
  \label{ampl:whiskas:bounds}  
\end{table}



We have formulated general problem using mathematical notation. Now we will proceed with model problem formulation using AMPL.

\subsubsection{Problem formulation using AMPL}

AMPL enables us to separate model definition and instance specific data. Usually we create three files: model, data and calling script. In the model file we define the data we need to have: sets and parameters, objective and constraints. Then in data file we populate the sets and parameters with the numbers for the particular instance of the problem. Both model and data files are loaded from calling script that may do some pre or post processing.

\paragraph{Model formulation}

First of all, we should define the sets for the ingredients and components of nutrition. Let's \todo{wyeliminowac let's} create file called \texttt{whiskas.mod}.

\begin{lstlisting}
set INGREDIENTS;
set COMPONENTS;
\end{lstlisting}

Using sets we can define the decision variables
\begin{lstlisting}
var Amount {INGREDIENTS} >= 0;
\end{lstlisting}

and parameters for the costs, components of nutrition contribution, restrictions and can size.

\begin{lstlisting}
param Cost {INGREDIENTS} >= 0;
param Contribution {INGREDIENTS, COMPONENTS} >= 0;

param Lower {COMPONENTS} default -Infinity;
param Upper {c in COMPONENTS} >= Lower[c], default Infinity;

param CanSize >= 0;

\end{lstlisting}

and the objective function

\begin{lstlisting}
minimize TotalCost : sum {i in INGREDIENTS} Cost[i] * Amount[i];
\end{lstlisting}

Note that, for the bounds that are not set, we assume \emph{±Infinity}. Additionally, we define that \emph{Cost} and \emph{Contribution} parameters should be non negative -- AMPL provides also verification of parameter data.


Finally, we may define constraints

\begin{lstlisting}
subject to MeetRequirements {c in COMPONENTS}:
  Lower[c] <= sum {i in INGREDIENTS} Contribution[i, c] * Amount[i] <= Upper[c];
  
subject to FullCan: 
  sum {i in INGREDIENTS} Amount[i] = CanSize;
\end{lstlisting}

\paragraph{Providing data}

Now we can provide the model with the data, let's call the file \texttt{whiskas.dat}.

First of all we need to provide sets we defined in previous paragraph.
\begin{lstlisting}
set INGREDIENTS := CHICKEN BEEF MUTTON RICE WHEAT GEL;
set COMPONENTS  := PROTEIN FAT FIBRE SALT;
\end{lstlisting}

Then we populate parameters with the data.
\begin{lstlisting}
param     Cost :=
  CHICKEN 0.013
  BEEF    0.008
  MUTTON  0.010
  RICE    0.002
  WHEAT   0.005
  GEL     0.001
;

param     Lower :=
  PROTEIN 8.0
  FAT     6.0
;

param     Upper :=
  FIBRE   2.0
  SALT    0.4
;

param CanSize := 100;

param Contribution :
          PROTEIN   FAT FIBRE  SALT :=
  CHICKEN   0.100 0.080 0.001 0.002 
  BEEF      0.200 0.100 0.005 0.005 
  MUTTON    0.150 0.110 0.003 0.007 
  RICE      0.000 0.010 0.100 0.002 
  WHEAT     0.040 0.010 0.150 0.008 
  GEL       0.0   0.0   0.0   0.0   
;
\end{lstlisting} 

\paragraph{Running AMPL}

Now we are ready to solve the model we formulated with AMPL. For that, let's create file called \texttt{whiskas.run}.

First of all we should reset AMPL environment in case that specific AMPL instance was solving other model before. We can do it with command \texttt{reset}. Then we load model using command \texttt{model} that takes model file name as an argument. Next, we load data file using command \texttt{data}. We also need to tell AMPL which solver we would like to be used. In our case we will use CPLEX (\texttt{option solver cplex}). Finally, we call solver and present results. The full file will look like as follows.

\begin{lstlisting}
reset;

model whiskas.mod;

data whiskas.dat;

option solver cplex;
solve;

display Amount;
\end{lstlisting}

Let's run AMPL and see what is the optimal solution.

\begin{lstlisting}
% ampl whiskas.run
CPLEX 12.4.0.1: optimal solution; objective 0.52
2 dual simplex iterations (0 in phase I)
Amount [*] :=
   BEEF  60
CHICKEN   0
    GEL  40
 MUTTON   0
   RICE   0
  WHEAT   0
;
\end{lstlisting}

In that case, it appears that it is cheapest to use beef and fill the rest of the can with gel.

\todo{przyklad ze schedulingiem}

\section{Conclusions}

This chapter presented basics of mathematical optimization. We discussed briefly history of optimiation, classification of optimization problems and introduced tools for mathematical optimization using computers. We also shown the example problem and corresponding model using AMPL.